\documentclass[10pt]{article} 
 
\usepackage[includeheadfoot, top=0.75in, bottom=0.5in, hmargin=0.75in]{geometry}
 
%\usepackage{amsmath, amsfonts, amssymb,epsfig,graphicx}
 
\usepackage{fancyhdr}
\usepackage{url}
\pagestyle{fancy}
\usepackage{setspace}
\usepackage{hyperref} 
\usepackage{xcolor}
\hypersetup{ 
     colorlinks=true, 
     linkcolor=blue, 
     filecolor=blue, 
     citecolor=black,       
     urlcolor=cyan} 
% \usepackage[colorlinks = true,
%             linkcolor = blue,
%             urlcolor  = blue,
%             citecolor = blue,
%             anchorcolor = blue]{hyperref}
% \newcommand{\changeurlcolor}[1]{\hypersetup{urlcolor=#1}}       

%\doublespacing
\singlespacing
%\onehalfspacing
 
\lhead{ASTR UN1903}
%\chead{}
\rhead{Fall 2022}
\lfoot{M. Sayeed}
\cfoot{\thepage}
\rfoot{Mon 6-9 pm}
 
\begin{document}
 
\begin{center}
{\huge ASTRONOMY LAB I}\\
%\medskip
%{\Large Wed 7 - 10pm}\\
%{\Large Pupin 1424}
\end{center}
 
\bigskip
 
 
\noindent \textbf{\normalsize Instructor:} {\normalsize Maryum Sayeed} \\ 
\textbf{\normalsize Email:} {\normalsize ms6341@columbia.edu}\\
\noindent \textbf{\normalsize Time/Location:} {\normalsize Mondays 6-9 PM} \\
\textbf{\normalsize Office hours:} {\normalsize by appointment, Pupin 1414}

 
\section*{Class Overview}
 
Welcome to the astronomy lab! The objectives of the lab are such that you will:
\begin{itemize}
\item Demystify the scientific method, develop critical thinking, and learn to apply scientific reasoning in your evaluation of information and arguments.
\item Develop a better sense of the scale of the Universe, an understanding of error and uncertainty in measurements, and other quantitative tools that will serve you beyond the scope of this lab.
\item Experience the scientific process of framing and asking a well defined question, how to gather data to test that question quantitatively, and how to communicate your work in text and speech to your peers.
\end{itemize}

\noindent In the end, the purpose of this class is to allow you to explore the Universe in a hands-on manner and to ask questions in a more personal setting than that afforded by large lecture classes. \\

\noindent There will be 10 labs throughout the semester, with no lab on November 7th (Election Day). There will be no homework assigned outside of our weekly 3-hour lab period. As a capstone project for this class, you will all give a short (10-minutes) presentation on an astronomy topic of your choice.

% There will be 10 labs this semester. There will be no work assigned outside of these sessions. For our final lab session, everyone is required to give a 10-minute presentation (followed by discussions for 5 min). List of topics related to astronomy and science in society will be rolled out. You are also welcome to suggest topics that may be of interest. Speak with your peers and me. Do not shy away from making mistakes and asking questions. Have fun!

\section*{Lab Materials}
 
Please bring the following to each lab session:
 
\begin{itemize}
\item \textbf{A lab notebook}: A bound notebook for your lab write-ups and reflections. I will collect your notebook at the end of every lab session and return it to you at the beginning of the next lab. 
 
\item \textbf{Writing/drawing tools}: Pen, pencil, eraser, ruler, etc. 
 
\item \textbf{Scientific calculator}:  A calculator capable of performing trigonometric functions, logarithms, exponents, roots, etc. A graphing calculator is not required. You may find \url{https: //www.desmos.com/scientific} or \url{https://www.wolframalpha.com/} useful if you wish to do calculations on your computer.

\item \textbf{Laptop}: Laptops will be a necessity for many of the labs. A limited number of laptops will be available for students who don’t have their own. Some labs may require you to watch short videos online, so headphones or earbuds are also recommended.
 
\end{itemize}

\section*{Grading}
\subsubsection*{Lab Write-ups}
\noindent Each lab will clearly denote what you should record in your write-ups for the lab. Lab responses can be recorded either in a bound physical notebook or in an electronic document. You may submit your work as a PDF to CourseWorks (strongly preferred), or you may hand in your lab notebook to me at the end of class, to be returned at the beginning of the following lab. All submissions will be due by midnight on the day of the lab; if this deadline poses a problem, please contact me. While I strongly recommend you work through labs with a partner, each of you should keep your own records. The purpose of the lab write-ups is for you to explain to me what you did during the lab, how you did it, and why you did it – I’m much more concerned with the reasoning behind your arguments than I am with the format of your submission. When doing quantitative problem solving (in any discipline), always show your work and clearly explain your thought process.\\

\noindent During each lab, I'll make clear what I'm looking for you to record in your lab notebooks (see below). At the end of each lab I will collect your notebooks and return them to you graded by the next lab period. If you are working with a partner, each of you should keep your own records. The whole goal of the write-ups is to explain to me \textit{what} you did during lab, \textit{how} you did
it, and \textit{why} you did it---I am much more concerned about being able to follow your argument/chain of reasoning than I am about the format. Additionally, at the end of each session I will ask that you write a few sentences reflecting on the lab we completed that day (ie. concepts you learned, things you were surprised by, questions you had, etc). For the final lab session, you will give a 10+5 minute presentation instead of submitting the write-up. Your lowest lab grade will be dropped when determining your final grade. \\

\noindent \textbf{Grading scheme:} Each lab will be graded out of 10 points based on clarity of writing (4 points), graphs, diagrams, and equations (3 points), and correctness (2 points).

\subsubsection*{Lab write-up guidelines:}
 
\begin{itemize}
% \setlength\itemsep{0.2em}
\setlength\parskip{0em}
\item[--] Each lab write-up should begin on a new page and have your name, your partner's name, lab title, and the date at the top.
\item[--] State specifically and in detail what your assumptions, methods, calculations, observations and conclusions are.
\item[--] Put a box around your final answer. However, note that the exact value is lesser important than the methods you used to get there (above).
\item[--] Always include units! Writing them in at each stage of calculations will help you keep track. Your units should be appropriate, e.g. don't measure the Sun in centimeters unless specifically asked to do so.
\item[--] Plots should have both axes labelled with units, and a legend or other indication of what each symbol/line represents
\item[--] Ensure your handwriting is legible. If I cannot read it easily, I won't be able to grade it and you won't get any points for that answer.
\end{itemize}
 
\subsubsection*{Participation}
\noindent As science is a collaborative discipline, you will be required to work with a partner for each lab. Your participation grade will be based on your contributions to your lab group, your class attendance, and your participation in class discussion. You are always encouraged to ask questions, regardless of content. You will also have the opportunity to record lingering questions in your lab write-ups.\\
 
\noindent Your final grade will be determined as follows:\\

\noindent 75\% Lab submissions*\\
15\% Final Presentation\\
10\% Participation 

\noindent*Your lowest lab grade will be dropped when determining your final grade.
% Backups: Convection, Indoor Telescope Lab

\section*{Policies}
 
\subsection*{Attendance}
 
By department policy, more than two unexcused (non-medical related) absences will result in automatic failure of the course. Please notify me if extenuating circumstances arise (family emergencies, serious illness, quarantine requirement, or religious holidays) and we will arrange a make-up lab.

\section*{Tentative Schedule}
\vspace{-1em}
\begin{table}[h!]
    \begin{tabular}{l|l|l}
        09/12 & Lab 1 & Units \& Orders of Magnitude\\% (Distance to the Sun)
        09/19 & Lab 2 & Jupiter's Moons \& Kepler's Laws \textit{(taught by Ben Cassese)}\\
        09/26 & Lab 3 & Height of Pupin\\
        10/03 & Lab 4 & The Moon\\
        10/10 & Lab 5 & Modeling the Earth-Sun System\\
        10/17 & Lab 6 & Light and the Sun\\
        10/24 & Lab 7 & Exoplanets\\
        10/31 & Lab 8 & \textit{Lab canceled}\\
        11/07 & ------ & --- NO LAB ---\\
        11/14 & Lab 9 & Observing Lab\\
        11/21 & Lab 10 & Astrobiology\\
        11/28 & Lab 11 & Prep for Final Presentations\\
        12/05 & Lab 12 & Final Presentations\\
    \end{tabular}
    \label{tab:schedule}
\end{table}

\vspace{-2em}
\section*{Accommodations}
Please speak with me if this course can be better adapted to your needs, without sacrificing the integrity of instruction. If you have an identified disability, I encourage you to register with the Office of Disability Services to ensure access to any necessary resources: \url{https://health.columbia.edu/services/register-disability-services}. Note that registration is confidential.
 
\section*{Academic Honesty}
Do not falsify data. Give credit to others' work. Do not present text verbatim from other sources as if it
were your own; do not otherwise plagiarize. Please ask if you are unsure what's acceptable. Academic honesty is taken very seriously and will lead to penalties if not abided to. For more information: \url{https://www.college.columbia.edu/academics/academicintegrity}
 
\section*{Mandatory Reporting}
I am required to report allegations of ``gender based misconduct, discrimination, or harassment" to
Columbia's administration. I am happy to listen and seek out resources (including confidential
counselors) on your behalf, but I cannot provide confidentiality myself.
 
\section*{Astronomy Events at Columbia}
Public lectures and observing sessions: \url{http://outreach.astro.columbia.edu/}

\section*{Other Concerns}
If there are any concerns you do not wish to raise directly with me, you may contact Prof. Laura Kay (lkay@barnard.edu), who supervises our astronomy labs.

\vspace*{\fill}
\centering\noindent Last updated: \today

\end{document}