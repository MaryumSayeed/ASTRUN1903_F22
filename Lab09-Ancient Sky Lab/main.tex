\documentclass[10pt]{article}% uses letterpaper by default

%---------- Uncomment one of them ------------------------------
\usepackage[includeheadfoot, top=.5in, bottom=.5in, hmargin=.5in]{geometry}

% \usepackage[a5paper, landscape, twocolumn, twoside,
%    left=2cm, hmarginratio=2:1, includemp, marginparwidth=43pt,
%    bottom=1cm, foot=.7cm, includefoot, textheight=11cm, heightrounded,
%    columnsep=1cm, dvips,  verbose]{geometry}
%---------------------------------------------------------------
\usepackage{fancyhdr}
\usepackage{verbatim}
\usepackage{url}
\pagestyle{fancy}
\usepackage{setspace}
%\doublespacing
\singlespacing
%\onehalfspacing
%\newcommand{\exercisename}{}

\chead{}
\rhead{Astronomy Lab}
\renewcommand{\rightmark}{}
\lfoot{}

\newcommand{\degrees}{\ensuremath{^\circ}}
\newcommand{\arcmin}{\ensuremath{'}}
\newcommand{\arcsec}{\ensuremath{"}}
\newcommand{\hours}{\ensuremath{^\mathrm{h}}}
\newcommand{\minutes}{\ensuremath{^\mathrm{m}}}
\newcommand{\seconds}{\ensuremath{^\mathrm{s}}}



\begin{document}


\begin{flushleft}
\begin{center}
\Large\textbf{Lab: Rules of Thumb \& Exploring the Night Sky}
\end{center}

\vspace{0.3in}

\large{In this lab, we will begin to learn how to navigate around the Night Sky.  First, we will come up with a handy (hehe) way of measuring the size of distant objects.  Then we will test that method out.  We will also learn how to use a planisphere to find constellations and then test how many of those we see outside.  Additionally, we will find out first hand just how many stars we DON'T see because of the lights of New York.}

\vspace{0.1in}

\large{\textbf{Part I --- Calibrate your hand to measure angles}}

\vspace{0.1in}

Sextants allow the precise measurement of angles, but not all of us
carry one in our back pocket (plus they take some practice). So you
are going to calibrate various parts of your hand to allow you to
measure angles.

\vspace{.1in}

Turn your head $90^\circ$ and extend your arm straight out from the
shoulder. Now extend your thumb like you're hitchhiking. This will be
one of your basic measuring devices. Now make a loose fist. There's
your other device.

\vspace{.1in}

\textbf{Record all your measurements and your calculations.  BE CLEAR.}
\vspace{.1in}

Measure the width of your thumb across the nail. Use the string to
measure the distance from your eye to your thumb at arm's length. Get
as accurate a measurement as possible. Check how much difference it
makes if you hold your arm in front of you versus to the side, or move
your head slightly. Calculate the angle subtended by your thumb in
degrees, and estimate the precision of your measurement.

\vspace{.1in}

Repeat this process for your fist (across the knuckles.)

\vspace{.1in}

You now have two ``devices'' for estimating angles that are always
with you.  Use both of them to measure the angular size of a few things in the room.  Pick 1 not too big objects and measure its angular size with both your fist and your thumb.  Then measure the distance from you to the object, and its actual size.  Use these to calculate the angular size as well, and check how close your measurments are to reality.  What percent error did you have?  Why might your measurement be inaccurate?

\vspace{0.1in}

\large{\textbf{Part II --- Making a Planisphere}}

\vspace{0.1in}

The \textbf{planisphere} is one of the handiest tools for a star-gazer. It was
refined in Greece during the 4th century by Hypatia, a notable mathematician
and astronomer of her day.

\vspace{0.1in}

\textbf{Construction}
%The print outs used in this lab can be found here: https://www.generationgenius.com/activities/sun-and-other-stars-activity-for-kids/
\vspace{0.1in}

1. Cut out both the top and bottom plate along the solid black edges. Cut other the middle of the bottom plate

2. Fold the bottom plate along the dashed lines

3. Insert the top plate into the bottom plate

\vspace{.1in}

\textbf{Using the Planisphere}

\vspace{.1in}

1. On a planisphere, the brightness of a star is represented by the size of its
dot. Usually a few dots sizes are used on the star map, each corresponding to a
different range of brightness, or \textbf{magnitude}. \textbf{What happens to
the number of stars you see of a given brightness as you consider dimmer and
dimmer magnitudes?} \textbf{Why does the star count depend on brightness like
this?} To simplify this problem, assume for a minute that all stars give off
the same amount of light.  Draw a diagram if you think it will aid your
explanation.

\vspace{.1in} 

2. Five of the planets in the sky are bright enough to outshine a lot of stars
shown on the planisphere. \textbf{Why aren't the planets mapped?} Hint: the
word ``planet'' is derived from the Greek word for ``wanderer''.

\vspace{.1in} 

3. Dial in the current time. What are the names of three bright stars overhead
now? When does Altair set today? When does it set a month from now?

\vspace{.1in} 

4. Why is the planisphere hole shaped the way it is?  Why does the pivot go through the North Star?  What aspect of the Earth's geometry/rotation is responsible for this?

\vspace{0.1in}

\large{\textbf{Part III --- Outdoor Observing}}


\vspace{0.1in} 

Using your fist, we'll try to measure how a star moves, and use that to estimate how quickly the Earth rotates. 

\vspace{0.1in} 

1. With your planisphere, orient yourself and identify as many constellations
as you can from the roof.  Write them down in your notebook.  \textbf{In what part of the sky is it easiest for you to see stars, and why?}

\vspace{0.1in} 

2. Find Polaris, the North Star. \textbf{What is its altitude?}

\vspace{0.1in} 

3. Pick a star near the Eastern or Western horizon.  Make sure it's bright enough that you can find it easily an hour from now.  Make a note on your planisphere if necessary. \textbf{Measure the altitude, first with your fist, then with your thumb.  Which do you think is more accurate? Why? Note the time.}

\vspace{0.1in} 


4. How do you expect the altitude of your star to change over the next hour?
Draw a diagram to explain.

\vspace{0.1in}

5. We will now spend some time with the telescope. If the weather allows it, we'll try to observe an object in the night sky (ideally Jupiter). If not, we'll tour the telescopes. If we could observe, what constellation does it look like our object belongs to? Draw a picture. If we can't observe, compare and contrast the big and little dome. What are the advantages and disadvantages to each telescope?


\vspace{0.1in}

\vspace{.1in}
\textbf{Finally, back to the altitude of your star:}

6. later on: note the time again. What is the altitude of the star now, measured just using your fist?

\vspace{0.1in}

7. From the two altitude measurements of the star near the horizon, and your
measurement of the altitude of Polaris, estimate how quickly the Earth rotates.
Use the method we discussed earlier.

\end{flushleft}


\end{document}
